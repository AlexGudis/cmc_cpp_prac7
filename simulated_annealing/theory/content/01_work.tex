\section*{Прикладная задача}
Дано $N$ независимых работ, для каждой работы задано время выполнения.
Требуется построить расписание выполнения работ без прерываний на $M$
процессорах. На расписании должно достигаться минимальное значение
\textit{критерия К1}.

\textbf{Критерий К1:} разбалансированность расписания.
\section*{Формальная постановка задачи}
\textbf{Дано:}
    \begin{itemize}
        \item $N$ -- количество работ.
        \item $M$ -- количество процессоров.
        \item $P = \{p_i\}$ -- множество работ, где $p_i = \{N_i, W_i\}$ и
            $i = \overline{1, N}$. $N_i$ -- номер $i$-й работы, $W_i$ -- её
            время выполнения.
        \item $PU = \{m_j\}$ -- множество процессоров, где $m_j$ -- $j$-й
            процессор, и $j = \overline{1, M}$.     
    \end{itemize}

    \textbf{Расписание}

    Расписанием является булева матрица $HP \in B^{N \times M}$, в которой $hm_{ij} \in \{0,1\}$, где $i \in \overline{1, N}$, а $j \in \overline{1, M}$.  
    Значение $hm_{ij} = 1$ означает, что работа с номером $i$ выполняется на процессоре с номером $j$,  
    а $hm_{ij} = 0$ --- что работа с номером $i$ не выполняется на процессоре с номером $j$.
    
    \textbf{Требуется}
    
    Построить расписание $HP^{N \times M}$, при котором будет минимизирован критерий.  
    При этом все задания будут выполнены на множестве процессоров $PU$ без прерываний,  
    с учётом ограниченных ресурсов, и не будет пересечений в использовании процессоров, т.е.
    
    \[
    \forall i \in \overline{1, N} \ \exists! \ j \in \overline{1, M} : hm_{ij} = 1
    \]
    
    что эквивалентно системе условий:
    
    \[
    \begin{cases}
    \displaystyle \sum_{i=1}^{N} \sum_{j=1}^{M} hp_{ij} = N, \\[8pt]
    \displaystyle \forall i \in \overline{1, N} \quad \sum_{j=1}^{M} hm_{ij} = 1.
    \end{cases}
    \]


    \textbf{Минимизируемый критерий:} \\
    Критерий разбалансированности расписания:
    \[
    K_1 = T_{\max} - T_{\min},
    \]
    где
    \[
    T_{\max} = \max_{j=1..M} T_j, \quad T_{\min} = \min_{j=1..M} T_j.
    \]
    
    \textbf{Требуется:} \\
    Найти такое корректное расписание $S^*$, что
    \[
    S^* = \arg\min_{S \in \Omega} K_1(S),
    \]
    где $\Omega$ --- множество всех корректных расписаний.

    \newpage
\section*{Исследование последовательного алгоритма}

    % ------------------ Время выполнения ------------------
    \begin{figure}[ht]
        \centering
        \begin{minipage}{0.32\textwidth}
            \centering
            \includegraphics[width=\textwidth]{/home/alexgud/study/msu/kurs4/prac/cmc_cpp_prac7/simulated_annealing/code/heatmaps/heatmap_Boltzmann_time.png}
            %\caption{Больцман}
            \label{fig:time1}
        \end{minipage}
        \hfill
        \begin{minipage}{0.32\textwidth}
            \centering
            \includegraphics[width=\textwidth]{/home/alexgud/study/msu/kurs4/prac/cmc_cpp_prac7/simulated_annealing/code/heatmaps/heatmap_Cauchy_time.png}
            %\caption{Коши}
            \label{fig:time2}
        \end{minipage}
        \hfill
        \begin{minipage}{0.32\textwidth}
            \centering
            \includegraphics[width=\textwidth]{/home/alexgud/study/msu/kurs4/prac/cmc_cpp_prac7/simulated_annealing/code/heatmaps/heatmap_Mixed_time.png}
            %\caption{Логарифмическое охлаждение}
            \label{fig:time3}
        \end{minipage}
        \caption{Время выполнения алгоритмов}
    \end{figure}
    
    % ------------------ Значения критерия ------------------
    \begin{figure}[ht]
        \centering
        \begin{minipage}{0.32\textwidth}
            \centering
            \includegraphics[width=\textwidth]{/home/alexgud/study/msu/kurs4/prac/cmc_cpp_prac7/simulated_annealing/code/heatmaps/heatmap_Boltzmann_k1.png}
            %\caption{Больцман}
            \label{fig:cost1}
        \end{minipage}
        \hfill
        \begin{minipage}{0.32\textwidth}
            \centering
            \includegraphics[width=\textwidth]{/home/alexgud/study/msu/kurs4/prac/cmc_cpp_prac7/simulated_annealing/code/heatmaps/heatmap_Cauchy_k1.png}
            %\caption{Коши}
            \label{fig:cost2}
        \end{minipage}
        \hfill
        \begin{minipage}{0.32\textwidth}
            \centering
            \includegraphics[width=\textwidth]{/home/alexgud/study/msu/kurs4/prac/cmc_cpp_prac7/simulated_annealing/code/heatmaps/heatmap_Mixed_k1.png}
            %\caption{Логарифмическое охлаждение}
            \label{fig:cost3}
        \end{minipage}
        \caption{Значения критерия для алгоритмов}
    \end{figure}
    
    \subsection*{Анализ времени выполнения}
    
    \begin{itemize}
    \item \textbf{Превышение порога 1 минуты:} 
    \begin{itemize}
    \item Конфигурации с $170\,000$ задач (все варианты процессоров)
    \item $100\,000$ задач с $150$--$200$ процессорами (законы Коши и смешанный)
    \end{itemize}
    
    \item \textbf{Максимальное время:} $88.05$ с ($170\,000$ задач, $50$ процессоров, закон Коши)
    
    \item \textbf{Ранжирование законов охлаждения по времени:}
    \begin{enumerate}
    \item Закон Коши (наибольшее время)
    \item Смешанный закон 
    \item Закон Больцмана (в $4$--$5$ раз быстрее)
    \end{enumerate}
    \end{itemize}
    
    \subsection*{Анализ качества решений (критерий K1)}
    
    \begin{itemize}
    \item \textbf{Лучшее качество:} Законы Коши и смешанный показывают сопоставимые результаты
    \item \textbf{Преимущество:} На $1$--$3\%$ лучше по сравнению с законом Больцмана
    \item \textbf{Пример:} Для $170\,000$ задач на $200$ процессорах:
    \begin{itemize}
    \item Коши/смешанный: $K1 \approx 43\,240$
    \item Больцман: $K1 \approx 44\,269$ ($+2.4\%$)
    \end{itemize}
    \end{itemize}
    
    \textbf{Вывод по результатам исследования последовательного алгоритма:} Закон Коши обеспечивает наилучшее качество расписаний ценой увеличения времени выполнения, что оправдывает его использование в параллельной реализации для обработки больших объемов данных.

    \newpage
\section*{Исследование параллельного алгоритма}
\begin{figure}[h]
    \centering
    \includegraphics[width=0.8\textwidth]{/home/alexgud/study/msu/kurs4/prac/cmc_cpp_prac7/simulated_annealing/code/output/parallel_metrics.png}
    %\caption{Описание изображения 2}
    \label{fig:image7}
\end{figure}

\textbf{Итог:} Анализ данных, полученных при $10\,000$ задачах и $50$ процессорах (усреднение по 5 запускам), показывает, что параллельная реализация демонстрирует значительное увеличение времени выполнения по сравнению с последовательной версией. Несмотря на это, наблюдается также и небольшой прирост производительности.
\end{document}